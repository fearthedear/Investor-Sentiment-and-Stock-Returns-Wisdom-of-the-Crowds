\begin{longtable}{@{}llll@{}}
\caption{Article Summary}\\
\label{article-overview}\\
\toprule
\textbf{Article} & \textbf{\begin{tabular}[c]{@{}l@{}}Study \\ (sentiment → returns)\end{tabular}} & \textbf{Effect size} & \textbf{\begin{tabular}[c]{@{}l@{}}Precision \\ (confidence interval)\end{tabular}} \\ \midrule
\multirow{3}{*}{\begin{tabular}[c]{@{}l@{}}Investor\\ sentiment and \\ stock returns, \\ Fisher and \\ Statman (2000)\end{tabular}} & \begin{tabular}[c]{@{}l@{}}Individual investors’ \\ sentiment → S\&P 500 \\ (1 month)\end{tabular} & \begin{tabular}[c]{@{}l@{}}1\% increase in \\ sentiment level → \\ 0.1\% decrease in S\&P \\ return next month\end{tabular} & \begin{tabular}[c]{@{}l@{}}Significant, 1\% level\\ T-statistic: -2.76\\ Adj. $R^2$ = 0.05 \\ Durbin-Watson = 1.91\end{tabular} \\ \cmidrule(l){2-4} 
 & \begin{tabular}[c]{@{}l@{}}Wall Street strategists’ \\ sentiment → S\&P 500 \\ (1 month)\end{tabular} & \begin{tabular}[c]{@{}l@{}}1\% increase in \\ sentiment level → \\ 0.24\% decrease in S\&P \\ return next month\end{tabular} & \begin{tabular}[c]{@{}l@{}}Significant (no level)\\ T-statistic: -2.41\\ Adj. $R^2$ = 0.03\\ Durbin-Watson = 2.14\end{tabular} \\ \cmidrule(l){2-4} 
 & \begin{tabular}[c]{@{}l@{}}Individual, Wall Street \\ investors and newsletter \\ writers aggregated \\ sentiment → S\&P 500  \\ (1 month) - correlation\end{tabular} & \begin{tabular}[c]{@{}l@{}}$R^2$ = 0.08 Sentiment \\ explains 8\% of the \\ variation in \\ S\&P 500 returns\end{tabular} & \begin{tabular}[c]{@{}l@{}}Significant, 1\% level\\ Adj. $R^2$ = 0.08\\ Durbin-Watson = 1.98\end{tabular} \\ \midrule
\multirow{4}{*}{\begin{tabular}[c]{@{}l@{}}The importance\\  of fear: \\ investor \\ sentiment and \\ stock market \\ returns, \\ Smales (2017)\end{tabular}} & \begin{tabular}[c]{@{}l@{}}VIX → S\&P 500\\ (1 month)\end{tabular} & \begin{tabular}[c]{@{}l@{}}1\% increase in \\ sentiment → \\ 0.86 \% decrease \\ in S\&P 500 return\end{tabular} & \begin{tabular}[c]{@{}l@{}}Significant: 1 \% level\\ AIC = 3.939\\ Durbin-Watson = 2.730\\ Wald F-stat = 146\end{tabular} \\ \cmidrule(l){2-4} 
 & \begin{tabular}[c]{@{}l@{}}VIX → large cap stocks\\ (1 month)\end{tabular} & \begin{tabular}[c]{@{}l@{}}1\% increase in \\ sentiment → \\ 0.014\% increase \\ in return\end{tabular} & \begin{tabular}[c]{@{}l@{}}Not significant: 10\% level\\ AIC = 5.704\\ Durbin-Watson = 2.004\\ Wald F-stat = 5.965\end{tabular} \\ \cmidrule(l){2-4} 
 & \begin{tabular}[c]{@{}l@{}}AAII → small cap stocks \\ (1 month)\end{tabular} & \begin{tabular}[c]{@{}l@{}}1\% increase in \\ sentiment → \\ 0.039 \% increase \\ in return\end{tabular} & \begin{tabular}[c]{@{}l@{}}Not significant: 10\% level\\ AIC = 6.438\\ Durbin-Watson = 2.020\\ Wald F-stat = 1.742\end{tabular} \\ \cmidrule(l){2-4} 
 & \begin{tabular}[c]{@{}l@{}}VIX → growth stocks \\ (1 month)\end{tabular} & \begin{tabular}[c]{@{}l@{}}1\% increase in \\ sentiment → 0.03 \% \\ decrease in return\end{tabular} & \begin{tabular}[c]{@{}l@{}}Not significant: 10\% level \\ AIC = 5.804\\ Durbin-Watson = 2.020\\ Wald F-stat = 3.866\end{tabular} \\ \midrule

\multirow{12}{*}{\begin{tabular}[c]{@{}l@{}}Investor \\ sentiment and\\ the near-term\\ stock market, \\ Brown and \\ Cliff (2002)\end{tabular}} & \begin{tabular}[c]{@{}l@{}}Kalman filter - \\sentiment of \\ professionals → \\ S\&P 500 (1 week)\end{tabular} & \begin{tabular}[c]{@{}l@{}}Combined lags\\ 1\% increase in \\ sentiment → 0.03\% \\ increase in return \end{tabular} & \begin{tabular}[c]{@{}l@{}}Significant:\\ 5\% level\end{tabular} \\  \cmidrule(l){2-4} 
 & \begin{tabular}[c]{@{}l@{}}Kalman filter \\(AAII, II) → \\ Russell 2000 Index\\ (1 week)\end{tabular} & \begin{tabular}[c]{@{}l@{}}Combined lags\\ 1\% increase in \\ sentiment → \\ 0.08\% decrease \\in return \end{tabular} & Not significant: 10\% level \\ \cmidrule(l){2-4} 
 
 & \multirow{3}{*}{\begin{tabular}[c]{@{}l@{}}Kalman filter \\(AAII, II) → \\ Largest quintile\\ NYSE/AMEX/\\NASDAQ\end{tabular}} 
 
 & \begin{tabular}[c]{@{}l@{}}Lag 1 month\\ 1\% increase in \\ sentiment → \\ 0.26\% increase\\ in return \end{tabular} & Not significant: 10\% level \\ \cmidrule(l){3-4} 

&  & \begin{tabular}[c]{@{}l@{}}Lag 2 months\\ 1\% increase in \\ sentiment → \\ 0.11\% decrease \\in return\end{tabular} & Not significant: 10\% level \\ \cmidrule(l){3-4} 

&  & \begin{tabular}[c]{@{}l@{}}Combined lags\\ 1\% increase in \\ sentiment → \\  0.8\% decrease \\in return\end{tabular} & Not significant: 10\% level \\ \cmidrule(l){2-4} 
 
 & \multirow{3}{*}{\begin{tabular}[c]{@{}l@{}}Kalman filter \\(AAII, II) → \\ 6-8 deciles\\ NYSE/AMEX/NASDAQ\end{tabular}} 
 
 & \begin{tabular}[c]{@{}l@{}}Lag 1 month\\ 1\% increase in \\ sentiment → \\ 0.58\% decrease \\in return \end{tabular} & Significant: 10\% level \\ \cmidrule(l){3-4} 
 &  & \begin{tabular}[c]{@{}l@{}}Lag 2 months\\ 1\% increase in \\ sentiment → \\ 0.003\% decrease\\ in return \end{tabular} & Not significant: 10\% level \\ \cmidrule(l){3-4} 
 &  & \begin{tabular}[c]{@{}l@{}}Combined lags \\ 1\% increase in \\ sentiment → \\ 0.01\% decrease \\in return \end{tabular} & Significant: 5\% level \\ 
 \midrule
\multirow{2}{*}{\begin{tabular}[c]{@{}l@{}}Investor \\ Sentiment in \\ the Stock \\ Market, \\ Baker and \\ Wurgler (2007)\end{tabular}} & \begin{tabular}[c]{@{}l@{}}Sentiment index from \\ 6 proxies → \\equal-weighted market \\index returns on \\ common stocks \\from CRSP database \\(1 month)\end{tabular} & \begin{tabular}[c]{@{}l@{}}Sentiment levels above \\ one standard deviation \\ from historical \\average → \\ -0.41\% stock returns.\end{tabular} & Significant: 10\% level \\ \cmidrule(l){2-4} 
 & \begin{tabular}[c]{@{}l@{}}Sentiment index from \\ 6 proxies → \\value-weighted \\ returns on common  stocks \\ from CRSP database\\ (1 month)\end{tabular} & \begin{tabular}[c]{@{}l@{}}Sentiment levels above \\ one standard deviation \\ from historical \\average → \\ -0.34\% stock returns.\end{tabular} & \begin{tabular}[c]{@{}l@{}}Significant:\\ 10\% level\end{tabular} \\ 
\midrule

\begin{tabular}[c]{@{}l@{}}Do investors’ \\ sentiment \\ dynamics \\ affect stock\\ returns? \\ Evidence from \\ the US economy, \\ Dergiades (2012)\end{tabular} & \begin{tabular}[c]{@{}l@{}}US investor sentiment \\ computed by Baker and \\ Wurgler (2007) → \\ US stock prices’ \\ index from 2005 \\ (1 month)\end{tabular} & \begin{tabular}[c]{@{}l@{}}Nonlinear causality \\tests by Hiemstra \& \\Jones  (1994) and Diks\\ \& Panchenko (2006)\\ show a rejection of null \\ hypothesis on 3 out \\ of 5 lags, implying a \\ nonlinear causal \\ relationship of \\ sentiment over \\ stock returns\end{tabular} & \begin{tabular}[c]{@{}l@{}}The results are significant \\ at the 5\% level for the \\ 1st and 5th lag, and at 10\% \\ for the 4th lag\end{tabular} \\ \midrule

\multirow{12}{*}{\begin{tabular}[c]{@{}l@{}}Investor \\ sentiment \\and stock \\returns: some \\international \\evidence \\ Schmeling (2009)\end{tabular}} 

& \begin{tabular}[c]{@{}l@{}}Consumer confidence → \\ CRSP aggregate \\stock market\\ (1 month)\end{tabular} & \begin{tabular}[c]{@{}l@{}} 1\% increase in \\ sentiment → 0.31\% \\ decrease in returns \end{tabular} & \begin{tabular}[c]{@{}l@{}}Significant: 5\% level\\Adj. $R^2$ = 0.06\end{tabular} \\ \cmidrule(l){2-4}

& \begin{tabular}[c]{@{}l@{}}Consumer confidence → \\ CRSP aggregate \\stock market\\ (6 month)\end{tabular} & \begin{tabular}[c]{@{}l@{}} 1\% increase in \\ sentiment → 0.31\% \\ decrease in returns \end{tabular} & \begin{tabular}[c]{@{}l@{}}Significant: 5\% level\\Adj. $R^2$ = 0.06\end{tabular} \\ \cmidrule(l){2-4}

& \begin{tabular}[c]{@{}l@{}}Consumer confidence → \\ CRSP aggregate \\stock market\\ (12 month)\end{tabular} & \begin{tabular}[c]{@{}l@{}} 1\% increase in \\ sentiment → 0.26\% \\ decrease in returns \end{tabular} & \begin{tabular}[c]{@{}l@{}}Significant: 5\% level\\Adj. $R^2$ = 0.11\end{tabular} \\ \cmidrule(l){2-4}

& \begin{tabular}[c]{@{}l@{}}Consumer confidence → \\ CRSP aggregate \\stock market\\ (24 month)\end{tabular} & \begin{tabular}[c]{@{}l@{}} 1\% increase in \\ sentiment → 0.20\% \\ decrease in returns \end{tabular} & \begin{tabular}[c]{@{}l@{}}Significant: 5\% level\\Adj. $R^2$ = 0.19\end{tabular} 

\\\midrule 
 
\multirow{12}{*}{\begin{tabular}[c]{@{}l@{}}When does \\ investor sentiment\\ predict stock \\returns? \\ Chung et al. \\(2012)\end{tabular}} 

& \begin{tabular}[c]{@{}l@{}}Baker and Wurgler \\ sentiment → US stock \\market returns\end{tabular} & \begin{tabular}[c]{@{}l@{}} 1\% increase in \\ sentiment → 0.71\% \\ increase in returns \end{tabular} & \begin{tabular}[c]{@{}l@{}}Significant: 1\% level\end{tabular} \\ \cmidrule(l){2-4}
 
& \begin{tabular}[c]{@{}l@{}}Baker and Wurgler \\ sentiment → US stock \\market returns\\(Recession)\end{tabular} & \begin{tabular}[c]{@{}l@{}} 1\% increase in \\ sentiment → 0.06\% \\ decrease in returns \end{tabular} & \begin{tabular}[c]{@{}l@{}}Not significant\\ at 10\% level\end{tabular} \\ \cmidrule(l){2-4}

& \begin{tabular}[c]{@{}l@{}}Baker and Wurgler \\ sentiment → US stock \\market returns\\(Expansion)\end{tabular} & \begin{tabular}[c]{@{}l@{}} 1\% increase in \\ sentiment → 1.03\% \\ increase in returns \end{tabular} & \begin{tabular}[c]{@{}l@{}}Significant: 1\% level\end{tabular} 

\\\midrule 

\multirow{12}{*}{\begin{tabular}[c]{@{}l@{}}Do measures \\ of investor \\sentiment predict \\returns?\\ Neal and \\Wheatley (1998)\end{tabular}} 

& \begin{tabular}[c]{@{}l@{}}Discount on \\closed ended funds → \\ Stock returns \\on NYSE-AMEX\\ (1 month)\end{tabular} & \begin{tabular}[c]{@{}l@{}} 1 standard deviation\\ increase in discount\\→ 0.36\% increase in \\returns\end{tabular} & \begin{tabular}[c]{@{}l@{}}Significant: 5\% level\end{tabular} \\ \cmidrule(l){2-4}

& \begin{tabular}[c]{@{}l@{}}Discount on \\closed ended funds → \\ Stock returns \\on NYSE-AMEX\\ (4 month)\end{tabular} & \begin{tabular}[c]{@{}l@{}} 1 standard deviation\\ increase in discount\\→ 0.52\% increase in \\returns\end{tabular} & \begin{tabular}[c]{@{}l@{}}Significant: 10\% level\end{tabular} \\ \cmidrule(l){2-4}

& \begin{tabular}[c]{@{}l@{}}Discount on \\closed ended funds → \\ Stock returns \\on NYSE-AMEX\\ (12 month)\end{tabular} & \begin{tabular}[c]{@{}l@{}} 1 standard deviation\\ increase in discount\\→ 0.91\% increase in \\returns\end{tabular} & \begin{tabular}[c]{@{}l@{}}Significant: 5\% level\end{tabular}

\\\bottomrule

\end{longtable}