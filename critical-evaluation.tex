\section{Critical Evaluation} \label{critical-evaluation}

\subsection{Introduction and Article Selection}
Even though the current literature on investor sentiment and stock returns employs a wide spectrum of methodologies and produces volatile results, the outcomes seem to incline towards the notion that investor sentiment negatively influences subsequent stock returns. This section presents an analysis of a comprehensive set of prior studies. The studies are carefully selected on the criteria of being strongly related to our hypothesis, the significance of their results and the appropriateness of methodologies used. Articles with wide range of different measures are analysed to gain insights on their applicability, validity and reliability.

\subsection{Hypotheses and Methodology}
As indicated above, even though most of the research mainly centres on investigating the influence of investor sentiment on the stock market returns, the exact methods and focus differ. Smales (2017), Neal and Wheatley (2001), Brown and Cliff (2002), Schmeling (2009) and Baker and Wurgler (2007) approach the research in a similar manner and employ the general hypothesis:  higher (lower) investor sentiment decreases (increases) stock returns. Some authors focus on individual investors only (Kumar and Lee 2002), test for non-linear relationship (Dergiades 2012) or the effect of recession and expansion on the relationship between sentiment and returns (Chung et al. 2012). Furthermore, there are differences in geographical location, the size of the firms analysed and time lag between measuring sentiment and stock returns. The following section summarizes the hypotheses and methodologies and compares the different approaches. The results of the studied literature are in Section~\ref{criticalsynthesis}, Critical Synthesis.

\paragraph{Geographical Location}
The articles focus mainly on US stock markets (NASDAQ and NYSE) where they also draw the stock returns data from. Schmeling (2009), however, investigates the effect of investor sentiment in 18 countries (including US) by employing consumer confidence as the sentiment proxy. This is in contrast with research conducted solely in the US, which mainly uses surveys or newsletters. Next to the general hypothesis, Schmeling (2009) also investigates hard-to-value stocks and the effect of market institution development and culture on the relationship between investor sentiment and stock returns. The paper tries to improve understanding of the source of investor sentiment (such as investor overreaction) by connecting Hofstede's (2001) findings. As such, the paper investigates whether cultural differences might play a role in behavioural biases. Chui et al. (2008) argue that investors in countries with high scores on collectivism are more prone to be influenced by the crowd and thus the connection between sentiment and returns is expected to be stronger.

\paragraph{Type of Firm}
Similar to Schmeling (2009), the second hypothesis of Smales (2017) aims to investigate whether small, hard-to-value firms react stronger to sentiment. This also applies to Brown and Cliff (2002), Fisher and Statman (2000) and Chung et al. (2012). The authors divide the companies based on book-to-market ratio and returns (Smales 2017), or the Center for Research in Security Prices' (CRSP) deciles and NYSE breakpoints. The papers differ in the use of CRSP deciles. CRSP deciles are companies divided into groups based on market capitalization.  The largest companies are in decile 1 and the smallest companies are in decile 10. For Fisher and Statman (2000), the small stocks are in deciles 9-10 whereas Brown and Cliff (2004) uses CRSP deciles 6-8. As largest stocks, they employ the largest quintile of NYSE/AMEX/NASDAQ. All the authors predict that stock returns of smaller firms are more likely to be influenced by investor sentiment as there is not enough information and thus the hard-to-value stocks are hard to arbitrage.

\paragraph{Recession and Expansion}
Chung et al. (2012) and Smales (2017) investigate the influence of recession and expansion of the economy state on the relationship between investor sentiment and stock returns. Both papers employ monthly sentiment measures and determine the state of economy based on data by the National Bureau of Economic Research (NBER). Chung et al. (2012) also use Markov switching models to determine what is the state of economy. Smales (2017) aims to investigate whether fear, represented by the Volatility Index (VIX), or confidence (Consumer Confidence Index) is better at predicting stock returns. Behavioural argumentation suggests that fear should be a better predictor, as people are more responsive to fear compared to confidence. This implies that during recession, investor sentiment should be more predictive of subsequent stock returns. 

\paragraph{Short-Term versus Long-Term}
Regarding the time frame and duration of the lags, the hypotheses of the authors test that the effect of investor sentiment is diminishing over time. This is a result of additional events affecting the market and changing the sentiment of investors. Subsequently, the investors are likely to adapt their strategy to this new information (Brown and Cliff 2005). Fisher and Statman (2000) investigate only one month lag periods whereas Schmeling (2009) looks at 1, 6, 12 and 24 months lag periods. Brown and Cliff (2002) also investigate weekly next to monthly periods. In multiple articles, the choice of the lag periods is influenced by the availability of investor sentiment data. This applies especially for international research conducted by Schmeling (2009). 

\paragraph{Individual vs Institutional}
Institutional and individual investors' investing strategies are diametrically different: the former as a profession, the latter mainly as a source of income alongside his professional job. This influences the behaviour of the distinct groups. Individual investors trade much less often, spend less time on investment analysis and rely on different sources of information than institutional investors do. Kumar and Lee (2003) aim to investigate the origins of the investor sentiment as they feel that most past studies tried to document only the mere existence of investor sentiment. 
\par
In their article, they find that individual investors tend to be more prevalent in the decile-9 stocks (small stocks – in the ninth decile of NYSE by market capitalization). Furthermore, they argue that the relative scarcity of publicly available information for smaller firms may induce individual investors to seek advice from investment newsletters. As a result, the activity around small stocks is largely influenced by the sentiment of investment newsletters (as the individual investors seek advice from this source and they are most prevalent in the segment of small stocks). In conclusion, individual investor sentiment tends to influence small stocks, value stocks, stocks with low institutional ownership and stocks with lower prices. In Fisher and Statman (2000), we can see that the sentiment of institutional investors and individual investors is not correlated at all (0.01). This underlines the information asymmetry, as well as the irrationality, which is likely to be exhibited to a larger extent by individual investors rather than Wall Street strategists.

\subsection{Research Strategies}
The different methodologies discussed above are implemented into very similar research strategies. The researchers conduct panel studies which are quasi-experimental. Panel studies are used when the variables are continuous and not binary (present/non-present). They have lower internal validity than experiments. However, in stock market research, experiments are not practically viable as it is impossible to halt trading and create an experimental environment. A panel study observes how the dependent and independent variables change over time. The causal effect can be observed in panel studies, but it is important to eliminate any influence of third variables on both the dependent and independent part. 
\par
Even though the internal validity of a panel study is lower than in experiment, it is higher than that of cross-sectional study, as there is a chronology of measuring independent and dependent variable. It is essential to determine the applicable time lag to observe the effects of the independent variable. The articles use time frames ranging from 1 week to 24 months and the results between these intervals differ.
\par
Exceptionally, Kumar and Lee (2003) tested their investor sample via robustness tests and are confident that the sample represents population studied. This increases the external validity of the research. Regarding stock returns, whole populations are analysed, increasing the external validity.

\subsection{Measurement, Validity and Reliability}
To assess the value of a set of studies, it is important to look at how researchers measure the variables and the resulting validity and reliability of the study.

\subsubsection{Investor Sentiment}
For investor sentiment measures, multiple databases are used. There are two main types of sentiment measures: direct, where investor directly shares their sentiment, and indirect, which use proxies to approximate the investor sentiment. Direct sentiment is usually collected through surveys, while indirect sentiment measures use proxies such as consumer confidence.

\paragraph{Direct Measures}
American Association of Individual Investors (AAII) conducts a monthly survey sent out to individual investors. Their responses are gathered and the results reflect the sentiment of individual investors on where the market will be in the next six months. The validity of AAII is high as the responses are directly collected and no proxies are used. The reliability can be considered high as the sample is always randomized and AAII has a large number of respondents. 
\par
Investors Intelligence (II) tracks the sentiment of 100 investment newsletters. The newsletters are created by semi-professional investors. As mentioned above, individual investors are highly influenced by the newsletters, especially for small and hard-to-value stocks where there is not enough information publicly available. The validity of this measure can be confirmed as it reflects the opinion (sentiment) of a large sample of investors. The reliability is questionable as the newsletter writers can become biased themselves based on external factors such as bandwagon behaviour (following opinions of others), macroeconomic matters or the evaluations can be inconsistent.
\par
Investors Intelligence and AAII correlate with a coefficient of 0.43, hinting that these two measures move in the same direction; however, they are not identical.

\paragraph{Indirect Measures}
Smales (2017) cites Whaley (2000), Simon and Wiggins (2001) and Giot (2005) who argue that VIX is a reliable contrarian indicator for investor sentiment. VIX is calculated by averaging the weighted prices of \^SPX (the index for S\&P 500) puts and calls over a wide range of strike prices and shows market volatility over the next 30 days. VIX is also known as the fear factor, as it represents higher expected volatility in the market. 
\par
Used in Smales (2017), The University of Michigan Consumer Sentiment Index picks a random sample of at least 500 consumers who are interviewed. The questionnaire has 50 questions and people answer on what will be their financial situation and near-term and long-term prospects for general economy. Schmeling (2009) suggests that when consumer sentiment is high (low) stock returns tend to be low (high) across multiple nations. Like the previous measures, Michigan Consumer Sentiment Index can also suffer from bias of the survey participants and low response ratio. The reliability is questionable as the sample always changes plus the answers of the people can be imprecise as the interviews are conducted over the phone and under time pressure.
\par
Schmeling (2009) uses consumer confidence as a proxy for the investor sentiment. Since he investigates stock markets internationally, this is the only available investor sentiment measure. Furthermore, it is available for reasonable periods of time and the countries tend to measure the consumer confidence consistently. Lemmon and Portniaguina (2006) provide an extensive analysis on why consumer confidence is a reliable proxy for investor sentiment.
\par
Baker and Wurgler (2007) construct their own sentiment measure by averaging six sentiment proxies: closed-end fund discount, de-trended log turnover, number of IPOs, first-day return on IPOs, dividend premium and equity share in new issues. These variables are chosen since the data is available and they are not skewed significantly due to macro-economic influences. They argue that while most of the studies take the bottom-up approach (using biases of individual investors which are aggregated), they apply a top-down approach (inferring about investor sentiment from macro-economic realities). This allows them to see which stocks are going to be most affected. Overall, Baker and Wurgler (2007) argue that the resulting sentiment measure successfully captures the sentiment as it lines up with the economic bubbles and crashes. Thus, this sentiment measure can be viewed as valid and reliable as it captures the investor sentiment reliably over time. However, they hold that both bottom-up and top-down approaches are vital for research. The bottom-up approach explores the variation of investor sentiment, whereas the top-down approach is likely to include market trends (bubbles and crashes) and patterns.
\par
Furthermore, Brown and Cliff (2002) use a variety of indirect measures: market performance, type of trading activity, derivatives variables, closed-end fund discount, net purchases of mutual funds, their available cash and first day returns (IPO) which is often associated with market tops. Most of these indirect measures are correlated with the increasing (monthly) investor survey data: cash of mutual funds (+), flow of their cash flows (+), close-ended fund discount (-), odd-lot (+) and short interest (+). Also, the market performance (+) is positively correlated with direct investor sentiment. Odd-lot and close-ended fund discount measure is also mentioned in Neal and Wheatley (1998). They take the odd-lot data from NYSE and discount from Wall Street Journal and Weisenberger's annual survey.
\newline
It seems that applying the aggregates of the measures of investor sentiment is the most reliable approach since some separate measures can suffer from missing data and biased opinions (e.g. II survey). Using an aggregate measure of the investor sentiment such as in Baker and Wurgler (2007) can lower the bias and provide richer data on investor sentiment.

\paragraph{Stock Returns}
Measuring the stock returns data is performed by obtaining stock prices from available databases, mainly CRSP. Schmeling (2009) investigates the relationship on international level and therefore draws data from databases in the respective countries. The measures of the stock prices, which are used to calculate the stock returns, are explicit and therefore valid and reliable. Some authors use logarithmic stock returns as the dependent variable.
\par
Most of the authors are regressing the stock returns on investor sentiment measures and additionally break down the stocks based on firm’s size, book/value ratio or growth or value nature of the firm. The following section evaluates the results obtained in the regressions and synthesizes across comparable domains, thereby creating a comprehensive review of the current state of research.