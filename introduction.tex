\section{Introduction and Theory}

The efficient market hypothesis, which is predominantly taught in university level finance courses, argues that stock markets are efficient in such a way that the current price of a stock always incorporates all known information. This implies that stocks are always priced at the monetary value of their actual worth (Fama 1965). Moreover, it means that reaping abnormal returns over continued time periods is not possible, because as soon as information affecting the value of a stock becomes available, it is already reflected in its price. While the efficient market hypothesis is an excellent way of introducing students to the way financial markets work, the history of stock markets quickly reveals that there are frequent occurrences where the theory falls short in explaining the movements of the market. The most vivid examples are booms, bubbles and crashes, such as the Great Crash of 1929 or the dot-com bubble of the late 1990s. The high volatility throughout such events illustrates that the stocks cannot constantly carry their actual, reasonable value. 
\par
Behavioural finance delivers an alternative view on the matter. It acknowledges that the part of the market which is still run by humans, not machines, is prone to human error and emotions such as investor sentiment. While the efficient market hypothesis was developed in the 1960s and has been popular since then, from the late 1980s onward a different perspective emerged and evolved, employing behavioural finance to find reasons behind extreme market movements. It seems the key event here was the stock market crash of 1987.  Delong, Shleifer, Summers, and Waldmann (1990) established the existence of investor sentiment, thereby paving the way for a series of studies on investor sentiment in the coming decades. 
\par
Investor sentiment can either be bearish, neutral or bullish - bearish is a weak investor sentiment; bullish is a strong investor sentiment, meaning that investors believe that the stock market will see price increases in the foreseeable future. Fisher \& Statman (2000) were among the first to establish that investor sentiment is commonly a contrarian indicator, meaning that a bullish sentiment is often followed by falling stock prices. The rationale behind this is that bullish sentiment often causes an overreaction by traders and lifts stock prices above the fundamental value of the stock (Barberis et. al 1998). When the market adjusts, stock prices fall. The inverted case of bearish sentiment and subsequent gains take place in similar, yet opposite fashion. Negative sentiment, built up in interaction with overreaction to negative news, causes immediate diminishing returns but in the short to medium-term drives prices upward during the adjustment.
\par
Going back to the roots of sentiment research, Baur et al. (1996) established that fundamentals, not sentiment, were behind the driving force behind market movements around the crash in 1987, but Shleifer \& Vishny (1997) found that betting against sentimental investors is risky and can be costly. Next, Neal \& Wheatley (1998) drew the connection between sentiment and stock returns by analysing the correlation of different sentiment measures and subsequent stock returns, and found that some of their implicit sentiment measures have a negative relationship to stock returns.
\par
In the 2000s, researcher developed a consensus about investor sentiment. The existence and the relation to stock returns were mostly undisputed. Thus, researchers focus shifted towards the circumstances under which the prediction value of sentiment regarding stock returns is especially valuable. Baker \& Wurgler (2007) state that stocks which are difficult to value or arbitrage are especially vulnerable to mispricing because of investor sentiment. This seems in line with the idea of placing sentiment as one of the reasons behind crashes: the dot-com bubble, for example, consisted of numerous overvalued start-up firms which are naturally hard to assess in their value, because valuations are calculated in terms of insecure future earnings. Schmeling (2007) added to this distinction behind stocks that not only the stock type makes a difference, but also the investors themselves. The sentiment of individual investors as well as investors in country-cultures with herd-like behaviour are more prone to overreaction and therefore have a higher sentiment and return correlation.
\par
The prediction value of sentiment is relevant for researchers as well as managers. If sentiment is the underlying cause of market crashes, sentiment knowledge may reduce extents of cash-destroying market movements. On a smaller scale, sentiment knowledge can support managers in assessing the risk of stock prices which are not based on fundamentals. Unfortunately, prior research inhibits some disagreements regarding the prediction value of sentiment for stock returns. Brown \& Cliff (2002) see little predictive power of investor sentiment for future returns, whereas Fisher \& Statman (2000) go as far as claiming that investor sentiment can be used in a tactical asset allocation programme. 
\par
To bring more clarity around this topic, our research investigates the following central hypothesis: Increases in investor sentiment are related to subsequent decreases in stock prices. With this hypothesis, the goal is to determine if stronger investor sentiment is followed by declining stock prices, as well as the strength of the decrease. This yields the following research question:

\vspace{10pt}
\begin{center}
\parbox{375pt}{ \centering
\textit{Is there a negative relationship between investor sentiment and stock market returns? If yes, how much prediction value does sentiment carry?}
}
\end{center}
\vspace{10pt}

While measuring stock returns is straightforward, correctly identifying and magnifying investor sentiment is a trickier undertaking. There are two distinct approaches. More common are sentiment surveys directly sent out to investors, such as the survey by American Association of Individual Investors (AAII). However, indirect attempts are utilized as well, such as proxying the consumer confidence index into investor sentiment (Schmeling 2009) or the first day returns during IPOs (Baker \& Wurgler 2006).  Investor sentiment is usually grouped by different types of investors, such as small-volume hobbyists or professional Wall Street investors. 
The dependent variable is stock market returns. Returns are the gains a specific stock achieved over a period of time, and can theoretically be measured on any stock, making the focal unit of our research the stock market in general. The main features of the research are summarised in Table~\ref{tab:research_features} below:

\begin{table}[h]
\centering
\begin{tabular}{>{\bfseries}l | r}
Focal Unit & Stock Market \\\hline
Theoretical Domain & All stock markets in all countries at all times\\\hline
Independent variable & Investor Sentiment\\\hline
Dependent variable & Stock returns\\\hline
Causal Relationship & Causal, probabilistic, negative
\end{tabular}
\caption{\label{tab:research_features}Research features}
\end{table}

\[
	\begin{tikzcd}
    Investor Sentiment \arrow{r}{-} & Stock Returns
	\end{tikzcd}
\]
\par 
In this paper, we will focus on the indicated unidirectional, negative relationship. Causality in investigating the relationship of investor sentiment and subsequent stock returns can be implied for multiple reasons: past papers, which established the existence of the relationship, applied time lags and applicable causality tests. Furthermore, there is economic theory behind the causality as well.  As mentioned, bullish sentiment often creates an overreaction by traders which lifts the stock price momentarily above its real value, but forces prices to fall after an appropriate time period has passed. For bearish sentiment the opposite chain reaction ensues.
\par
These are the cornerstones of our research. In the next chapter of this paper, Section~\ref{critical-evaluation},  we will introduce a set of articles with similar research objectives and thereby create an overview of the current state of research. We will analyse the quality of these papers in terms of their research strategies and variable measurement techniques. Section~\ref{criticalsynthesis} finalizes the literature review by comparing the effect sizes of the articles, i.e. examining the strength of the sentiment-returns relationship. Section~\ref{research-proposal-section} lays out the structure of our own research and, afterwards, we present its results in Section~\ref{results-section}. Finally, we provide a detailed discussion of these results and present our conclusions in Section~\ref{discussion-section}.